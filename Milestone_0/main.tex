\documentclass[12pt, a4paper]{article}
\usepackage[utf8]{inputenc}
\usepackage{hyperref}
\usepackage[left=1.00in, right=1.00in, top=1.00in, bottom=1.00in]{geometry}
\title{CS348 Project Milestone 0}
\author{Abdullah Bin Assad\and Chandana Sathish \and Lukman Mohamed \and Vikram Subramanian \and Dhvani Patel}
\begin{document}
\maketitle

\section*{Our Application}
We are creating an interactive web app to help users explore the crime data in their neighborhood. This could be especially useful for people who plan to assess how safe a neighborhood is before moving there, while walking home at odd times, or people who are just curious about this kind of data. We are presently experimenting with a number of different cities and data sets. Our primary data sets are taken from the Toronto Police Open Data Portal (\url{https://data.torontopolice.on.ca/pages/open-data}). Another batch of data sets we are looking at come from the New York Open Data (\url{https://data.cityofnewyork.us/Public-Safety/NYC-crime/qb7u-rbmr}). 
\section*{Platform and Tech Stack}
Our interface will consist of an interactive web-app. We will be using GCP to host our database and run a MySQL instance. This setup is ideal and efficient for building a web app such as ours. GCP and MySQL are modern, scalable, efficient and flexible. We will be using Node.js with an Express.js server as our back-end to interact with the database. For the front-end, we will be using the React library. Thus, our web application code mostly consists of JavaScript. Members of our group have experience with Node.js and React which gives us an excellent basis for our project.

\end{document}
