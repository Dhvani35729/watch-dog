\documentclass{article}
\usepackage[utf8]{inputenc}
\usepackage{hyperref}
\title{CS348 Project Milestone 0}
\author{}
\begin{document}
\maketitle

\section*{Our Application}
We are creating an interactive web app to help users gain access to crime data in their neighborhood. This could be especially useful for people who plan to assess how safe a neighborhood is before moving into a new place, while walking home at odd times or people who are just curious about this kind of data. We are presently experimenting with the number of different cities and datasets we are pulling from. Our primary datasets are taken from the Toronto Police Open Data Portal (\url{https://data.torontopolice.on.ca/pages/open-data}). Another batch of datasets we are looking at come from the  the New York Open Data (\url{https://data.cityofnewyork.us/Public-Safety/NYC-crime/qb7u-rbmr}). 
\section*{Platform and Tech stack}
Our interface will consist of an interactive web-app. We will be using GCP to host our database and run a MySQL instance. This setup is ideal and efficient for building a web app such as ours. GCP and MySQL are modern, scalable, efficient and flexible. We will be using [Django/Node.js] for our back-end and to interact with the database. We sought permission from the instructors as ORM’s are next specifically mentioned in the project documents (cs348\_s20\_project, project-notes). Such permission can be seen granted in Piazza (\url{https://piazza.com/class/k9u44trc5yd6iq?cid=154}). Members of our groups have experience with [Django/Node.js] which gives us an excellent basis for our project.

\end{document}
